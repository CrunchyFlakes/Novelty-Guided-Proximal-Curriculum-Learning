% This is a variation of the NeurIPS'24 format
\documentclass{article}
\usepackage[final]{adrl}

\usepackage[utf8]{inputenc} % allow utf-8 input
\usepackage[T1]{fontenc}    % use 8-bit T1 fonts
\usepackage{url}
\usepackage{xcolor}     
\usepackage{spreadtab}

\newcommand{\TODO}[1][]{\textcolor{red}{\bf [TODO]}}

\title{ADRL Project Proposal Self-Grading}
\author{Theo Retisch \\ \url{github.com/theo/adrl_project}}

\begin{document}

\maketitle

You should score your own work in each of these categories from 0-10 points per question. 0 in this case is "did not do this at all" and 10 is "did this perfectly".
The tables will autocomplete your score for each category and you can then decide from these scores how you would grade yourself in total.

You don't have to completely follow our points in grading yourself, but as an orientation: 100-95\% would be equivalent to 1.0, 95-90\% to 1.3 and so on. 
We would furthermore weight the categories as follows: 15\% research understanding, 30\% idea \& implementation, 30\% empirical evaluation and 25\% writing. 
This is also only a guideline and you don't need to follow it exactly.

\section{Research Understanding}

\begin{spreadtab}{{tabular}{|l|c|c||c|}}
\hline
    @ Question & @ Points given & @ Weight & @ Point Total \\
    \hline
    \hline
    @ I described the task setting of my problem well & 0 & 0.3 & b2 * c2\\
    \hline
    @ I referenced the most important work within that task setting & 0 & 0.3 & b3 * c3\\
    \hline
    @ At least 3 of my references were not in the lecture slides & 0 & 0.3 & b4 * c4\\
    \hline
    @ I provided references on similar approaches with other objectives & 0 & 0.1 & b5 * c5\\
    \hline
    \hline
    @ Result & b2+b3+b4+b5 & c2+c3+c4+c5 & d2+d3+d4+d5 \\
\hline
\end{spreadtab}

\section{Idea \& Implementation}

\begin{spreadtab}{{tabular}{|l|c|c||c|}}
\hline
    @ Question & @ Points given & @ Weight & @ Point Total \\
    \hline
    \hline
    @ My idea is small enough to explore in this project & 0 & 0.2 & b2 * c2\\
    \hline
    @ I can express my idea in 1-3 research questions & 0 & 0.3 & b3 * c3\\
    \hline
    @ My idea is based on concepts from the lecture & 0 & 0.1 & b5 * c5\\
    \hline
    @ My idea is new in the setting I consider & 0 & 0.1 & b5 * c5\\
    \hline
    @ My implementation can be compared to state-of-the-art approaches & 0 & 0.2 & b4 * c4\\
    \hline
    @ I provide good quality code for my idea & 0 & 0.1 & b5 * c5\\
    \hline
    \hline
    @ Result & b2+b3+b4+b5+b6+b7 & c2+c3+c4+c5+c6+c7 & d2+d3+d4+d5+d6+d7 \\
\hline
\end{spreadtab}

\textbf{Open Question:} How do you score the long-term impact of your idea? What does it contribute to the research landscape overall?

\textbf{Answer:}
\TODO{}

\section{Empirical Evaluation}

\begin{spreadtab}{{tabular}{|l|c|c||c|}}
\hline
    @ Question & @ Points given & @ Weight & @ Point Total \\
    \hline
    \hline
    @ My experiments all aim to answer my research questions & 0 & 0.3 & b2 * c2\\
    \hline
    @ I follow the experimental standards from the lecture & 0 & 0.3 & b3 * c3\\
    \hline
    @ I ensure reproducibility as much as I can & 0 & 0.3 & b4 * c4\\
    \hline
    @ All my results are shown with confidence intervals & 0 & 0.1 & b5 * c5\\
    \hline
    \hline
    @ Result & b2+b3+b4+b5 & c2+c3+c4+c5 & d2+d3+d4+d5 \\
\hline
\end{spreadtab}

\textbf{Open Question:} Are there parts of your research questions you were not able to address empirically? Which ones?

\textbf{Answer:}
\TODO{}

\section{Writing}

\begin{spreadtab}{{tabular}{|l|c|c||c|}}
\hline
    @ Question & @ Points given & @ Weight & @ Point Total \\
    \hline
    \hline
    @ My writing is clear and understandable & 0 & 0.2 & b2 * c2\\
    \hline
    @ I explain all concepts someone from the RL1 lecture might not know & 0 & 0.2 & b3 * c3\\
    \hline
    @ I have a reference or empirical insight for each claim & 0 & 0.3 & b4 * c4\\
    \hline
    @ Where necessary, I use figures to illustrate my ideas & 0 & 0.1 & b5 * c5\\
    \hline
    @ The spelling and formatting is good and supports readability & 0 & 0.2 & b5 * c5\\
    \hline
    \hline
    @ Result & b2+b3+b4+b5+b6 & c2+c3+c4+c5+c6 & d2+d3+d4+d5+d6 \\
\hline
\end{spreadtab}

\textbf{Open Question:} Do you think your write-up accurately reflects the work you did? Why or why not?

\textbf{Answer:}
\TODO{}

\section{Final Grade}
\textbf{Summary Statement:} 
\emph{Add 2-3 sentences about strengths and weaknesses of the project and why you think the grade is correct.}

\TODO{}


\textbf{Final grade:} \TODO{}
\end{document}
